% ********** Rozdział 4 **********
\chapter{Prezentacja i omówienie wyników}
\label{sec:chapter4}

\section{Uruchomienie aplikacji i konfiguracja środowiska}

Aplikacja uruchamiana jest lokalnie przy użyciu wbudowanego serwera Laravel oraz skonfigurowanego połączenia z bazą danych. Po poprawnej konfiguracji środowiska użytkownik otrzymuje dostęp do strony startowej aplikacji.

\begin{figure}[H]
    \centering
    \includegraphics[width=0.75\linewidth]{img/R_logowanie.png}
    \caption{Ekran startowy aplikacji (ekran logowania)}
    \label{fig:placeholder}
\end{figure}

\section{Proces logowania i autoryzacji użytkownika}

Logowanie do aplikacji odbywa się wyłącznie poprzez integrację z platformą GitHub z wykorzystaniem protokołu OAuth 2.0.

\begin{figure}[H]
    \centering
    \includegraphics[width=0.5\linewidth]{img/R_github_auth.png}
    \caption{Ekran logowania OAuth 2.0 na platformie GitHub}
    \label{fig:placeholder}
\end{figure}

Po poprawnej autoryzacji użytkownik zostaje automatycznie zalogowany i przekierowany do panelu głównego.

\begin{figure}[H]
    \centering
    \includegraphics[width=1\linewidth]{img/R_projects_list.png}
    \caption{Lista projektów}
    \label{fig:placeholder}
\end{figure}

\section{Zarządzanie zespołami i rolami użytkowników}

Użytkownik może tworzyć zespoły, zapraszać członków oraz przypisywać im role zgodne z modelem Scrum. Dostęp do poszczególnych funkcjonalności aplikacji zależy od roli przypisanej w danym zespole.


\begin{figure}[H]
    \centering
    \includegraphics[width=1\linewidth]{img/R_team_view.png}
    \caption{Widok zespołu}
    \label{fig:placeholder}
\end{figure}


\begin{figure}[H]
    \centering
    \includegraphics[width=1\linewidth]{img/R_team_name_edit.png}
    \caption{Karta edycji nazwy projektu}
    \label{fig:placeholder}
\end{figure}

\section{Integracja projektów z platformą GitHub}

Aplikacja umożliwia połączenie projektu z repozytorium GitHub w celu pobierania zgłoszeń i pull requestów. Synchronizacja danych odbywa się ręcznie z poziomu interfejsu użytkownika.


\begin{figure}[H]
    \centering
    \includegraphics[width=1\linewidth]{img/R_sync_github.png}
    \caption{Interfejs synchronizacji (przykładowo z repozytorium microsoft/vscode)}
    \label{fig:sync_github}
\end{figure}

Po synchronizacji projektu z repozytorium możliwy jest wybór issue lub pull request w zadaniu.

\begin{figure}[H]
    \centering
    \includegraphics[width=0.5\linewidth]{img/R_github_task.png}
    \caption{Wybór Issue z GitHub pod zadaniem}
    \label{fig:github_task}
\end{figure}

\section{Backlog produktu}

Backlog produktu prezentuje listę wszystkich zadań oczekujących na realizację. Użytkownik może dodawać, edytować oraz zmieniać kolejność zadań zgodnie z priorytetami projektu.

\begin{figure}[H]
    \centering
    \includegraphics[width=0.75\linewidth]{img/R_project_backlog.png}
    \caption{Backlog produktu}
\end{figure}

\subsection{Dzielenie historyjek na mniejsze jednostki}

\begin{figure}[H]
    \centering
    \includegraphics[width=0.6\linewidth]{img/R_subtask_splitting.png}
    \caption{Interfejs dzielenia historyjki w podzadania}
    \label{fig:subtask_splitting}
\end{figure}

\section{Sprinty i tablica}

Sprinty są prezentowane w formie tablicy, podzielonej na kolumny odpowiadające statusom realizacji zadań. Takie podejście umożliwia szybki podgląd postępu prac zespołu.

\begin{figure}[H]
    \centering
    \includegraphics[width=0.75\linewidth]{img/R_tasks_grid.png}
    \caption{Tablica zadań}
\end{figure}

\subsection{Współpraca zespołowa w czasie rzeczywistym}

Zmiany w zadaniach sprintu są synchronizowane w czasie rzeczywistym pomiędzy użytkownikami dzięki wykorzystaniu WebSocketów. Pozwala to na równoczesną pracę wielu członków zespołu bez konieczności odświeżania strony.

\section{Dokumentacja, historia aktywności i retrospektywy}

Projekt zawiera sekcję dokumentów oraz historię aktywności, umożliwiającą śledzenie zmian w projekcie. Dodatkowo po zakończeniu sprintu dostępne są retrospektywy wspierające proces ciągłego doskonalenia zespołu.

\begin{figure}[H]
    \centering
    \includegraphics[width=1\linewidth]{img/R_docs.png}
    \caption{Strona dokumentów projektowych}
\end{figure}

\section{Testy manualne}

Testy manualne zostały przeprowadzone w końcowej fazie implementacji oraz na bieżąco w trakcie prac rozwojowych. Obejmowały one ręczne sprawdzenie działania interfejsu użytkownika, poprawności przepływu danych oraz zgodności funkcji z założeniami projektowymi. Weryfikowano m.in. możliwość tworzenia i edycji zadań, logikę planowania sprintów, autoryzację ról oraz responsywność aplikacji na różnych rozdzielczościach. Testy manualne pozwoliły wychwycić błędy interakcji i braki walidacji, które zostały skorygowane przed finalnym wdrożeniem.

Dodatkowo, finalna wersja aplikacji została przekazana do testów trzem niezależnym osobom, które nie brały udziału w jej tworzeniu. Osoby te otrzymały konta testowe z różnymi poziomami uprawnień (m.in. administrator, użytkownik zespołu) i zestaw scenariuszy do samodzielnego wykonania. Testujący oceniali poprawność działania kluczowych funkcji, przejrzystość interfejsu oraz intuicyjność obsługi. Zgłoszone przez nich uwagi dotyczyły głównie ergonomii oraz drobnych nieścisłości w formularzach i komunikatach systemowych, które zostały poprawione.

\section{Testy jednostkowe i stabilność aplikacji}

Testy jednostkowe zostały wykorzystane do weryfikacji poprawności działania modeli, polityk oraz kluczowej logiki aplikacji. Ich zastosowanie zwiększa niezawodność systemu oraz ułatwia dalszy rozwój projektu.

\begin{figure}
    \centering
    \includegraphics[width=0.7\linewidth]{img/R_tests.png}
    \caption{Diagram wyników testów jednostkowych}
\end{figure}

% ********** Koniec rozdziału **********

% ********** Rozdział 5 **********
\chapter{Wnioski}
\label{sec:chapter5}

Zrealizowany system spełnił swoją rolę jako narzędzie wspierające codzienną organizację pracy zespołu i zarządzanie projektami w sposób przejrzysty i uporządkowany. Dzięki przemyślanej strukturze interfejsu oraz implementacji kluczowych mechanizmów Scrum, aplikacja okazała się praktyczna i intuicyjna w użyciu - zarówno dla pojedynczych użytkowników, jak i zespołów.

Umożliwia przejrzyste planowanie sprintów, wygodne zarządzanie backlogiem oraz bieżące śledzenie postępów zadań - wszystko w spójnym i intuicyjnym interfejsie. System dobrze sprawdził się w scenariuszach typowych dla pracy zespołów programistycznych i może być skutecznym wsparciem w realizacji projektów prowadzonych w duchu zwinnego wytwarzania oprogramowania.

Zastosowane technologie - Laravel po stronie backendu i React w warstwie interfejsu - zapewniły solidne podstawy techniczne oraz elastyczność niezbędną przy budowie nowoczesnej aplikacji webowej. Dzięki Inertia.js możliwe było zintegrowanie logiki serwera z dynamicznym interfejsem użytkownika, bez potrzeby budowy osobnego API. Projektowana architektura oraz podział ról pozwoliły osiągnąć wysoką czytelność i spójność kodu.

Testy przeprowadzone zarówno przez autora, jak i trzech niezależnych użytkowników potwierdziły poprawność działania kluczowych funkcji oraz dobrą ergonomię systemu. Opinie testujących przyczyniły się do wprowadzenia kilku istotnych poprawek w zakresie wygody obsługi i komunikacji wizualnej.

\section*{Dalszy rozwój}

Chociaż aplikacja spełnia swoje główne założenia, otwarta jest również przestrzeń na przyszłe rozszerzenia:

\begin{itemize}
    \item \textbf{Zaawansowane zarządzanie rolami i uprawnieniami} - możliwość definiowania niestandardowych ról i przypisywania im precyzyjnych uprawnień.
        
    \item \textbf{Integracje z innymi systemami logowania} - np. Google lub poprzez e-mail.
    
    \item \textbf{Motyw ciemny i personalizacja UI} - możliwość dostosowania wyglądu interfejsu do preferencji użytkownika, np. trybu ciemnego.
        
    \item \textbf{Eksport danych} - zapis projektów do plików CSV lub PDF.

    \item \textbf{Tryb offline aplikacji} - pozwalający na pracę bez stałego połączenia z siecią i późniejszą synchronizację.
\end{itemize}

Z uwagi na modularną strukturę i przejrzystą architekturę systemu, projekt można łatwo rozwijać i dostosowywać do bardziej zaawansowanych potrzeb. Aplikacja może być również dobrą podstawą dla dalszych prac badawczo-rozwojowych lub wdrożeń produkcyjnych.

\bigskip

System w obecnej formie oferuje pełnoprawne wsparcie dla zespołów realizujących projekty programistyczne w sposób zwinny, a jego dalszy rozwój może jeszcze bardziej zwiększyć jego praktyczną wartość.

% ********** Koniec rozdziału **********
