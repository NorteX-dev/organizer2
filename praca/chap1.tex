% ********** Rozdział 1 **********

\chapter{Wstęp}
\label{sec:chapter1}

\section{Motywacja pojęcia tematu}
\label{sec:chapter1:motywacja}

Główną motywacją podjęcia tematu jest stworzenie projektu wysoce dostępnego dla wszystkich, w szczególności małych zespołów freelancingowych i organizacji non-profit. Osiągane jest to przez otwarcie kodu źródłowego na platformie open-source GitHub i przez zastosowanie modelu nieodpłatnej aplikacji. Dodatkowo stosowana licencja MIT umożliwia dowolną modyfikację i odsprzedaż. Otwarte źródło tego projektu zapewni jednocześnie dalsze życie po zakończeniu wsparcia przez oryginalnego autora, ponieważ każda osoba może pobrać, modyfikować i rozwijać. Motywacją projektu nie jest wykonanie kompletnie unikalnego i nowego na rynku rozwiązania.

Agile opiera się na iteracyjnym dostarczaniu wartości i ciągłej adaptacji do zmian, dlatego w projektach o niepewnych lub ewoluujących wymaganiach zwykle przewyższa metody planistyczne (np. waterfall), które zakładają stabilny zakres i utrudniają szybkie reagowanie na informację zwrotną.

\section{Cel pracy}
\label{sec:chapter1:cel}
Celem pracy jest zaprojektowanie i realizacja aplikacji internetowej wspomagającej zarządzanie projektami programistycznymi prowadzonymi zgodnie z filozofią metodyki Agile Scrum. Opracowane rozwiązanie ma na celu umożliwiać wygodną organizację projektów, planowanie iteracji, kontrolę postępów prac oraz komunikację zespołu projektowego. Podstawowym założeniem jest zapewnienie przejrzystego i przyjaznego użytkownikom interfejsu, który usprawni realizację zadań wykonywanych podczas sprintów.
Kluczowym aspektem jest także założenie otwartości źródłowej wynikowego projektu. Oprogramowanie ma posiadać otwartą licencję MIT oraz być udostępnione na platformie GitHub jako projekt FOSS.

\section{Teza pracy}

Opracowanie i wdrożenie aplikacji internetowej wspomagającej zarządzanie projektami zgodnie z metodyką Agile Scrum pozwala na zwiększenie efektywności organizacji pracy zespołów projektowych, usprawnienie komunikacji oraz poprawę kontroli nad realizacją sprintów, jednocześnie oferując małym zespołom i organizacjom non-profit realną, bezpłatną alternatywę wobec komercyjnych rozwiązań.

Tezą pracy jest również założenie, że dobór technologii zastosowanych w projekcie - w szczególności Laravel i PHP po stronie serwera oraz React po stronie klienta - wraz z pozostałymi narzędziami zastosowanymi w projekcie jest adekwatny do wymagań systemu. Wybrane technologie umożliwią sprawną implementację, utrzymanie i dalszy rozwój aplikacji przy zachowaniu dobrej wydajności oraz czytelnej architektury.

\section{Zakres pracy}
\label{sec:chapter1:zakres}

Zakres pracy obejmuje przegląd literatury dotyczącej metodyk zwinnych (agile), analizę istniejących narzędzi i oprogramowań wspierających zespoły projektowe, opracowanie wymagań funkcjonalnych i niefunkcjonalnych systemu, przygotowanie projektu bazy danych, wybór technologii, na której będzie oparta aplikacja, implementację aplikacji i wymaganych zależności oraz przeprowadzenie testów własnych i zautomatyzowanych, potwierdzających poprawność działania rozwiązania.


% ********** Koniec rozdziału **********
